\documentclass[11pt,oneside,a4paper]{article}

\usepackage{mathtext}
\usepackage[T1,T2A]{fontenc}
\usepackage[utf8]{inputenc}
\usepackage[english,russian]{babel}

\title{Лабораторная работа 3. Использование автоматичиских генераторов и анализаторов ANTLR и Bison}
\author{Никита Ященко, М3338}
\date{4 Мая 2016}

\begin{document}

\maketitle

\section{Задание}
\subsection{Вариант 8.}
 
Выберите подмножетсво TeX и напишите его конвертор в HTML.
При необходимости используйте MathML.

\subsection{Пример}

\begin{verbatim}
$a_i = b_i + x^2$
\end{verbatim}

Или целиком этот документ.

\section{Ход работы}

Выбрано подмножество TeX, включающее в себя функции, секции, блоки и немного математических операций.
Для отображения математических символов и операций ипользуется MathJax.

\section{Примеры}

\subsection{Числа и переменные}

$a$, $2$, $\pi$

\subsection{Операции}

$2 + 2 * 2$, $1 + 2 * 3 / 4$, $a + b * c$
 
\subsection{Отношения}

$a < b \le c$

$d = e \ne f$

$g > h \ge i$

\subsection{Математика}

Add $a$ squared and $b$ squared to get $c$ squared.
Or, using a more mathematical approach: $a^2 + b^{2} = c^2$.

100 $m^{3}$ of water.

$x_{1} + x_{1} = 2 * x_{1}$

$\lim_{x \to 6}{x} = 6$

$\lim_{n \to \infty}{1/n} = \frac{\pi^2}{6} * 0$

$\sum_{k=1}^{5}{k} = \frac{60}{2}/2$

${5^{\frac{1}{x}} = A}_{y} - \frac{1}{2 + \frac{3}{4 + \frac{5}{c}}} + \sum_{i=1}^{5}{i^{2*i}}$

$\int_{\frac{\pi}{2}}^{\infty}$

\subsection{Дополнительное задание}

Минимум

$\min X$

Минимум с нижним индексом

$\min_{subscript} x$

limits для суммы и интеграла

$\sum\limits_{k=1}^{5}{k} = 15$

$\int\limits_{\frac{\pi}{2}}^{\infty}$
 
\end{document}
